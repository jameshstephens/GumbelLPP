% Content for Approximate Gumbel Last Passage Percolation
% This file contains the mathematical content of the blueprint

\chapter{Introduction}

This project formalizes the convergence properties of Last Passage Percolation (LPP) models with approximate Gumbel distributions. We establish that LPP with $N$-approximate Gumbel weights converges to the same GUE Tracy-Widom distribution as exact Gumbel LPP, provided that $N$ grows appropriately with the grid size.

The main result uses a coupling argument between exact and approximate Gumbel distributions, combined with perturbation bounds to control the difference between the two models.

\chapter{Grid Paths and Last Passage Percolation}

\section{Basic Definitions}

\begin{definition}[Grid Point]
  \label{def:grid_point}
  \lean{GridPoint}
  \leanok
  A \emph{grid point} is an element of $\mathbb{N}^2$.
\end{definition}

\begin{definition}[Edge]
  \label{def:edge}
  \lean{Edge}
  \leanok
  An \emph{edge} is a pair of grid points.
\end{definition}

\begin{definition}[Up-Right Edge]
  \label{def:up_right}
  \lean{is_up_right}
  \leanok
  An edge $(p, q)$ where $p = (x, y)$ and $q = (x', y')$ is \emph{up-right} if either:
  \begin{itemize}
    \item $x' = x + 1$ and $y' = y$ (right step), or
    \item $x' = x$ and $y' = y + 1$ (up step).
  \end{itemize}
\end{definition}

\begin{definition}[Grid Path]
  \label{def:grid_path}
  \lean{GridPath}
  \leanok
  A \emph{grid path} is a list of edges.
\end{definition}

\begin{definition}[Valid Path]
  \label{def:valid_path}
  \lean{IsValidPath}
  \leanok
  \uses{def:up_right}
  A path is \emph{valid} from point $p$ to point $q$ if it is a sequence of connected up-right edges starting at $p$ and ending at $q$.
\end{definition}

\begin{lemma}[Paths Exist]
  \label{lem:paths_nonempty}
  \lean{paths_nonempty}
  \leanok
  \uses{def:valid_path}
  For any $m, n \in \mathbb{N}$, there exists at least one valid path from $(0, 0)$ to $(m, n)$.
\end{lemma}

\begin{definition}[LPP Value]
  \label{def:lpp_value}
  \lean{LPP_value}
  \leanok
  \uses{def:valid_path}
  Given a weight function $w : \text{Edge} \to \mathbb{R}$ and endpoints $(m, n)$, the \emph{Last Passage Percolation value} is:
  \[
    \text{LPP}_w(m, n) = \max_{\pi \in \text{Paths}(0, 0; m, n)} \sum_{e \in \pi} w(e)
  \]
  where the maximum is taken over all valid paths from $(0, 0)$ to $(m, n)$.
\end{definition}

\chapter{Gumbel and Exponential Distributions}

\section{The Gumbel Distribution}

\begin{definition}[Gumbel CDF]
  \label{def:gumbel_cdf}
  \lean{gumbel_cdf}
  \leanok
  The \emph{Gumbel cumulative distribution function} is:
  \[
    F_{\text{Gumbel}}(x) = \exp(-e^{-x})
  \]
\end{definition}

\begin{lemma}[Gumbel CDF is Continuous]
  \label{lem:gumbel_cdf_continuous}
  \lean{gumbel_cdf_continuous}
  \leanok
  \uses{def:gumbel_cdf}
  The Gumbel CDF is continuous on $\mathbb{R}$.
\end{lemma}

\begin{definition}[Gumbel Grid]
  \label{def:gumbel_grid}
  \lean{IsGumbelGrid}
  \leanok
  \uses{def:gumbel_cdf}
  A random field $Y : \text{Edge} \to \Omega \to \mathbb{R}$ is a \emph{Gumbel grid} if:
  \begin{enumerate}
    \item The random variables $\{Y_e\}_{e \in \text{Edge}}$ are independent, and
    \item For each edge $e$ and $x \in \mathbb{R}$, $\mathbb{P}(Y_e \leq x) = F_{\text{Gumbel}}(x)$.
  \end{enumerate}
\end{definition}

\section{Transformation to Exponential Distribution}

\begin{lemma}[Gumbel Measure of Singletons is Zero]
  \label{lem:gumbel_measure_singleton_zero}
  \lean{gumbel_measure_singleton_zero}
  \leanok
  \uses{def:gumbel_cdf, lem:gumbel_cdf_continuous}
  If $Y$ is a Gumbel random variable, then for any $y \in \mathbb{R}$:
  \[
    \mathbb{P}(Y = y) = 0
  \]
\end{lemma}

\begin{proof}
  This follows from the continuity of the Gumbel CDF. For any continuous CDF $F$, the probability of a singleton is the difference $F(y) - F(y^-)$, which is zero when $F$ is continuous.
\end{proof}

\begin{lemma}[Gumbel Probability of Complement]
  \label{lem:gumbel_prob_ge}
  \lean{gumbel_prob_ge}
  \leanok
  \uses{def:gumbel_cdf, lem:gumbel_measure_singleton_zero}
  If $Y$ is a Gumbel random variable, then:
  \[
    \mathbb{P}(Y \geq y) = 1 - F_{\text{Gumbel}}(y)
  \]
\end{lemma}

\begin{lemma}[Gumbel to Exponential Transformation]
  \label{lem:gumbel_to_exp_cdf}
  \lean{gumbel_to_exp_cdf}
  \leanok
  \uses{def:gumbel_cdf, lem:gumbel_prob_ge}
  If $Y$ is a Gumbel random variable, then $\exp(-Y)$ has the exponential distribution with rate 1. Specifically, for $x \geq 0$:
  \[
    \mathbb{P}(\exp(-Y) \leq x) = 1 - e^{-x}
  \]
\end{lemma}

\begin{proof}
  \uses{lem:gumbel_prob_ge}
  For $x > 0$, we have:
  \begin{align*}
    \mathbb{P}(\exp(-Y) \leq x) &= \mathbb{P}(-Y \leq \log x) \\
    &= \mathbb{P}(Y \geq -\log x) \\
    &= 1 - F_{\text{Gumbel}}(-\log x) \\
    &= 1 - \exp(-\exp(\log x)) \\
    &= 1 - e^{-x}
  \end{align*}
\end{proof}

\begin{lemma}[Exponential Grid from Gumbel]
  \label{lem:exp_neg_Y_is_exp_grid}
  \lean{exp_neg_Y_is_exp_grid}
  \leanok
  \uses{def:gumbel_grid, lem:gumbel_to_exp_cdf}
  If $Y$ is a Gumbel grid, then $E_e = \exp(-Y_e)$ forms a grid of independent exponential random variables with rate 1.
\end{lemma}

\chapter{The Coupling Construction}

\section{Approximate Gumbel Distribution}

\begin{definition}[Approximate Gumbel CDF]
  \label{def:approx_gumbel_cdf}
  \lean{f_approx_gumbel}
  \leanok
  For $N \geq 1$, the \emph{$N$-approximate Gumbel CDF} is:
  \[
    F_N(x) = \begin{cases}
      \left(1 - \frac{e^{-x}}{N}\right)^N & \text{if } x > -\log N \\
      0 & \text{otherwise}
    \end{cases}
  \]
\end{definition}

\section{The Coupling Function}

\begin{definition}[Coupling Function $h_N$]
  \label{def:h_function}
  \lean{h}
  \leanok
  For $N \geq 1$, define:
  \[
    h_N(x) = -\log\left(N \cdot \left(1 - e^{-e^{-x}/N}\right)\right) - x
  \]
\end{definition}

\begin{lemma}[Convexity and Bounds for $h_N$]
  \label{lem:h_properties}
  \lean{lemma_1}
  \leanok
  \uses{def:h_function}
  For $N \geq 1$, the function $h_N : \mathbb{R} \to \mathbb{R}$ satisfies:
  \begin{enumerate}
    \item $h_N$ is convex on $\mathbb{R}$,
    \item $0 < h_N(x) \leq \frac{e^{-x}}{N}$ for all $x \in \mathbb{R}$,
    \item $\frac{e^{-x}}{3N} \leq h_N(x)$ for all $x > 0$.
  \end{enumerate}
\end{lemma}

\begin{proof}
  The proof uses calculus to verify convexity by showing the second derivative is non-negative. The upper bound follows from Taylor expansion of the exponential and logarithm. The lower bound for $x > 0$ uses the inequality $1 - e^{-t} \geq t - \frac{t^2}{2} + \frac{t^3}{6}$ for $t \geq 0$.
\end{proof}

\begin{theorem}[Coupling Identity]
  \label{thm:coupling_identity}
  \lean{coupling_identity}
  \leanok
  \uses{def:gumbel_cdf, def:approx_gumbel_cdf, def:h_function}
  For $N \geq 1$ and $y \in \mathbb{R}$:
  \[
    F_{\text{Gumbel}}(y) = F_N(h_N(y) + y)
  \]
\end{theorem}

\begin{proof}
  Direct calculation shows both sides equal $\exp(-e^{-y})$.
\end{proof}

\section{LPP Definitions}

\begin{definition}[Gumbel LPP]
  \label{def:T_gumbel}
  \lean{T_Gumbel}
  \leanok
  \uses{def:lpp_value, def:gumbel_grid}
  For a Gumbel grid $Y$, define:
  \[
    T_{\text{Gumbel}}(n) = \text{LPP}_Y(n, n)
  \]
\end{definition}

\begin{definition}[Approximate Gumbel LPP]
  \label{def:T_approx}
  \lean{T_Approx}
  \leanok
  \uses{def:lpp_value, def:h_function}
  For a Gumbel grid $Y$ and $N \geq 1$, define:
  \[
    T_{\text{Approx}}^N(n) = \text{LPP}_{Y + h_N(Y)}(n, n)
  \]
  where the weights are $w_e = Y_e + h_N(Y_e)$ for each edge $e$.
\end{definition}

\begin{definition}[Exponential LPP]
  \label{def:L_exp}
  \lean{L_Exp}
  \leanok
  \uses{def:lpp_value}
  For a grid $E$ of exponential random variables:
  \[
    L_{\text{Exp}}(n) = \text{LPP}_E(n, n)
  \]
\end{definition}

\chapter{Perturbation Analysis}

\begin{lemma}[Perturbation Bounds]
  \label{lem:perturbation_bounds}
  \lean{lemma_2}
  \leanok
  Let $\Pi$ be a finite nonempty set, and $S_A, S_B : \Pi \to \mathbb{R}$ be functions. Suppose $\pi^*$ maximizes $S_A$. Define:
  \begin{itemize}
    \item $M_A = S_A(\pi^*)$
    \item $M_{A+B} = \max_{\pi \in \Pi} (S_A(\pi) + S_B(\pi))$
    \item $m_B = \min_{\pi \in \Pi} S_B(\pi)$
    \item $M_B = \max_{\pi \in \Pi} S_B(\pi)$
  \end{itemize}
  Then:
  \[
    m_B \leq S_B(\pi^*) \leq M_{A+B} - M_A \leq M_B
  \]
\end{lemma}

\begin{proof}
  The first inequality is immediate. For the second, note that $M_{A+B} \geq S_A(\pi^*) + S_B(\pi^*) = M_A + S_B(\pi^*)$. For the third, let $\pi'$ maximize $S_A + S_B$. Then:
  \[
    M_{A+B} - M_A = S_A(\pi') + S_B(\pi') - S_A(\pi^*) \leq S_B(\pi') \leq M_B
  \]
  where we used $S_A(\pi^*) \geq S_A(\pi')$.
\end{proof}

\section{Coupling Bounds}

\begin{theorem}[Coupling Upper Bound]
  \label{thm:coupling_upper_bound}
  \lean{coupling_upper_bound}
  \leanok
  \uses{def:T_gumbel, def:T_approx, def:L_exp, lem:h_properties}
  For $N \geq 1$, a Gumbel grid $Y$, and $n \in \mathbb{N}$:
  \[
    T_{\text{Approx}}^N(n) - T_{\text{Gumbel}}(n) \leq \frac{1}{N} \cdot L_{\text{Exp}}(n)
  \]
  where $L_{\text{Exp}}(n)$ is computed with weights $E_e = e^{-Y_e}$.
\end{theorem}

\begin{proof}
  \uses{lem:perturbation_bounds}
  Let $\pi^*$ be the maximizing path for $T_{\text{Gumbel}}$. By Lemma \ref{lem:h_properties}, for each edge $e$:
  \[
    h_N(Y_e) \leq \frac{e^{-Y_e}}{N}
  \]
  Summing over the maximizing path for $T_{\text{Approx}}^N$ and using Lemma \ref{lem:perturbation_bounds} gives the result.
\end{proof}

\begin{theorem}[Coupling Lower Bound]
  \label{thm:coupling_lower_bound}
  \lean{coupling_lower_bound}
  \leanok
  \uses{def:T_gumbel, def:T_approx, lem:h_properties}
  For $N \geq 1$, a Gumbel grid $Y$, and $n \in \mathbb{N}$:
  \[
    2n \cdot h_N\left(\frac{T_{\text{Gumbel}}(n)}{2n}\right) \leq T_{\text{Approx}}^N(n) - T_{\text{Gumbel}}(n)
  \]
\end{theorem}

\begin{proof}
  Let $\pi^*$ be the geodesic for $T_{\text{Gumbel}}$, which has length $2n$. By Jensen's inequality applied to the convex function $h_N$:
  \[
    \frac{1}{2n} \sum_{e \in \pi^*} h_N(Y_e) \geq h_N\left(\frac{1}{2n} \sum_{e \in \pi^*} Y_e\right) = h_N\left(\frac{T_{\text{Gumbel}}(n)}{2n}\right)
  \]
  The result follows since $T_{\text{Approx}}^N(n) \geq \sum_{e \in \pi^*} (Y_e + h_N(Y_e))$.
\end{proof}

\chapter{Convergence Properties}

\section{Convergence Definitions}

\begin{definition}[Convergence in Probability to Zero]
  \label{def:converges_in_prob_zero}
  \lean{ConvergesInProbZero}
  \leanok
  A sequence of random variables $\{X_n\}$ \emph{converges in probability to zero} if:
  \[
    \forall \varepsilon > 0, \quad \mathbb{P}(|X_n| > \varepsilon) \to 0 \text{ as } n \to \infty
  \]
\end{definition}

\begin{definition}[Convergence in Probability to a Constant]
  \label{def:converges_in_prob_const}
  \lean{ConvergesInProbConst}
  \leanok
  A sequence of random variables $\{X_n\}$ \emph{converges in probability to $c$} if:
  \[
    \forall \varepsilon > 0, \quad \mathbb{P}(|X_n - c| > \varepsilon) \to 0 \text{ as } n \to \infty
  \]
\end{definition}

\section{Known Results (Axiomatized)}

The following properties capture known results from the literature that we assume as axioms:

\begin{definition}[Exact Gumbel Convergence Property]
  \label{def:exact_gumbel_convergence}
  \lean{ExactGumbelConvergenceProperty}
  For a Gumbel grid and appropriate constants $C_g, \sigma_g > 0$:
  \[
    \frac{T_{\text{Gumbel}}(n) - C_g \cdot n}{\sigma_g \cdot n^{1/3}} \xrightarrow{d} F_{\text{GUE}}
  \]
  where $F_{\text{GUE}}$ is the GUE Tracy-Widom distribution.
\end{definition}

\begin{definition}[Time Constant for Gumbel LPP]
  \label{def:time_constant_gumbel}
  \lean{TimeConstantGumbelProperty}
  There exists a constant $D_\ell > 0$ such that:
  \[
    \frac{T_{\text{Gumbel}}(n)}{n} \xrightarrow{\mathbb{P}} D_\ell
  \]
\end{definition}

\begin{definition}[Time Constant for Exponential LPP]
  \label{def:time_constant_exp}
  \lean{TimeConstantExpProperty}
  There exists a constant $D_L > 0$ such that:
  \[
    \frac{L_{\text{Exp}}(n)}{n} \xrightarrow{\mathbb{P}} D_L
  \]
\end{definition}

\section{Slutsky's Theorem}

\begin{theorem}[Slutsky Upper Bound]
  \label{thm:slutsky_upper_bound}
  \lean{slutsky_upper_bound}
  \leanok
  For random variables $X, Y$ and constants $r, \varepsilon$:
  \[
    \mathbb{P}(X + Y \leq r) \leq \mathbb{P}(X \leq r + \varepsilon) + \mathbb{P}(|Y| > \varepsilon)
  \]
\end{theorem}

\begin{theorem}[Slutsky Lower Bound]
  \label{thm:slutsky_lower_bound}
  \lean{slutsky_lower_bound}
  \leanok
  For random variables $X, Y$ and constants $r, \varepsilon$:
  \[
    \mathbb{P}(X \leq r - \varepsilon) \leq \mathbb{P}(X + Y \leq r) + \mathbb{P}(|Y| > \varepsilon)
  \]
\end{theorem}

\begin{theorem}[Slutsky's Theorem for CDFs]
  \label{thm:slutsky_cdf}
  \lean{slutsky_cdf}
  \leanok
  \uses{def:converges_in_prob_zero, thm:slutsky_upper_bound, thm:slutsky_lower_bound}
  Suppose $X_n \xrightarrow{d} F$ (convergence in distribution to a continuous CDF $F$) and $Y_n \xrightarrow{\mathbb{P}} 0$. Then:
  \[
    X_n + Y_n \xrightarrow{d} F
  \]
\end{theorem}

\begin{proof}
  Fix $r \in \mathbb{R}$ and $\varepsilon > 0$. By Theorems \ref{thm:slutsky_upper_bound} and \ref{thm:slutsky_lower_bound}:
  \begin{align*}
    \mathbb{P}(X_n \leq r - \varepsilon) - \mathbb{P}(|Y_n| > \varepsilon) &\leq \mathbb{P}(X_n + Y_n \leq r) \\
    &\leq \mathbb{P}(X_n \leq r + \varepsilon) + \mathbb{P}(|Y_n| > \varepsilon)
  \end{align*}
  Taking limits and using continuity of $F$ gives the result.
\end{proof}

\section{Auxiliary Convergence Lemmas}

\begin{lemma}[Product Convergence]
  \label{lem:converges_in_prob_mul_zero}
  \lean{converges_in_prob_mul_zero}
  \leanok
  \uses{def:converges_in_prob_zero, def:converges_in_prob_const}
  If $Y_n \xrightarrow{\mathbb{P}} c$ and $a_n \to 0$, then $a_n \cdot Y_n \xrightarrow{\mathbb{P}} 0$.
\end{lemma}

\begin{lemma}[Deterministic Factor Limit]
  \label{lem:deterministic_factor_limit}
  \lean{deterministic_factor_limit}
  \leanok
  For $\alpha > 2/3$:
  \[
    \frac{n}{\lfloor n^\alpha \rfloor \cdot n^{1/3}} \to 0 \text{ as } n \to \infty
  \]
\end{lemma}

\begin{proof}
  We have:
  \[
    \frac{n}{\lfloor n^\alpha \rfloor \cdot n^{1/3}} \leq \frac{2n}{n^\alpha \cdot n^{1/3}} = 2n^{2/3 - \alpha}
  \]
  Since $\alpha > 2/3$, the exponent is negative and the limit is zero.
\end{proof}

\chapter{Main Theorem}

\begin{theorem}[Approximate Gumbel Convergence: Critical Threshold at $\alpha = 2/3$]
  \label{thm:main}
  Assume the properties in Definitions \ref{def:exact_gumbel_convergence}, \ref{def:time_constant_gumbel}, and \ref{def:time_constant_exp}. Let $N_n = \lfloor n^\alpha \rfloor$ for some $\alpha > 0$. For a sequence of Gumbel grids $Y^{(n)}$:

  \begin{enumerate}
    \item \textbf{(Convergence for $\alpha > 2/3$)} If $\alpha > 2/3$, then for any $r \in \mathbb{R}$:
    \[
      \mathbb{P}\left(\frac{T_{\text{Approx}}^{N_n}(n) - C_g \cdot n}{\sigma_g \cdot n^{1/3}} \leq r\right) \to F_{\text{GUE}}(r)
    \]
    as $n \to \infty$. That is, the approximate Gumbel LPP converges to the same GUE Tracy-Widom distribution as the exact Gumbel LPP.

    \item \textbf{(Divergence for $\alpha < 2/3$)} If $\alpha < 2/3$, then the fluctuations diverge:
    \[
      \frac{T_{\text{Approx}}^{N_n}(n) - C_g \cdot n}{\sigma_g \cdot n^{1/3}} \xrightarrow{\mathbb{P}} +\infty
    \]
    as $n \to \infty$. More precisely, for any $M > 0$:
    \[
      \mathbb{P}\left(\frac{T_{\text{Approx}}^{N_n}(n) - C_g \cdot n}{\sigma_g \cdot n^{1/3}} > M\right) \to 1
    \]
  \end{enumerate}

  Thus $\alpha = 2/3$ represents a sharp threshold: the approximation parameter $N$ must grow faster than $n^{2/3}$ for the limiting distribution to remain Tracy-Widom GUE.
\end{theorem}

\begin{proof}
  \uses{thm:coupling_upper_bound, thm:coupling_lower_bound, thm:slutsky_cdf, lem:converges_in_prob_mul_zero, lem:deterministic_factor_limit, lem:h_properties}

  By Theorems \ref{thm:coupling_upper_bound} and \ref{thm:coupling_lower_bound}:
  \[
    2n \cdot h_{N_n}\left(\frac{T_{\text{Gumbel}}(n)}{2n}\right) \leq T_{\text{Approx}}^{N_n}(n) - T_{\text{Gumbel}}(n) \leq \frac{1}{N_n} L_{\text{Exp}}(n)
  \]

  Dividing by $\sigma_g n^{1/3}$:
  \[
    \frac{2n \cdot h_{N_n}(T_{\text{Gumbel}}(n)/(2n))}{\sigma_g n^{1/3}} \leq \frac{T_{\text{Approx}}^{N_n}(n) - T_{\text{Gumbel}}(n)}{\sigma_g n^{1/3}} \leq \frac{L_{\text{Exp}}(n)}{N_n \sigma_g n^{1/3}}
  \]

  \textbf{Case 1: $\alpha > 2/3$.}

  By Definition \ref{def:time_constant_exp}, $L_{\text{Exp}}(n)/n \xrightarrow{\mathbb{P}} D_L$. We have:
  \[
    \frac{L_{\text{Exp}}(n)}{N_n \sigma_g n^{1/3}} = \frac{L_{\text{Exp}}(n)}{n} \cdot \frac{n}{N_n \sigma_g n^{1/3}}
  \]

  By Lemma \ref{lem:deterministic_factor_limit}, $n/(N_n n^{1/3}) \to 0$ when $\alpha > 2/3$. Therefore by Lemma \ref{lem:converges_in_prob_mul_zero}:
  \[
    \frac{T_{\text{Approx}}^{N_n}(n) - T_{\text{Gumbel}}(n)}{\sigma_g n^{1/3}} \xrightarrow{\mathbb{P}} 0
  \]

  By Definition \ref{def:exact_gumbel_convergence}:
  \[
    \frac{T_{\text{Gumbel}}(n) - C_g n}{\sigma_g n^{1/3}} \xrightarrow{d} F_{\text{GUE}}
  \]

  Applying Slutsky's Theorem (Theorem \ref{thm:slutsky_cdf}) gives:
  \[
    \frac{T_{\text{Approx}}^{N_n}(n) - C_g n}{\sigma_g n^{1/3}} \xrightarrow{d} F_{\text{GUE}}
  \]

  \textbf{Case 2: $\alpha < 2/3$.}

  By the lower bound from Theorem \ref{thm:coupling_lower_bound} and Lemma \ref{lem:h_properties}, for $x > 0$:
  \[
    h_{N_n}(x) \geq \frac{e^{-x}}{3N_n}
  \]

  By Definition \ref{def:time_constant_gumbel}, $T_{\text{Gumbel}}(n)/n \xrightarrow{\mathbb{P}} D_\ell$ where $D_\ell > 0$. Therefore $T_{\text{Gumbel}}(n)/(2n) \xrightarrow{\mathbb{P}} D_\ell/2 > 0$, which means for large $n$, $T_{\text{Gumbel}}(n)/(2n)$ is bounded away from zero with high probability. Thus:
  \[
    \frac{2n \cdot h_{N_n}(T_{\text{Gumbel}}(n)/(2n))}{\sigma_g n^{1/3}} \geq \frac{2n \cdot e^{-T_{\text{Gumbel}}(n)/(2n)}}{3N_n \sigma_g n^{1/3}} \geq \frac{C \cdot n}{N_n n^{1/3}} = C \cdot n^{2/3 - \alpha}
  \]
  for some constant $C > 0$ (with high probability). When $\alpha < 2/3$, the exponent $2/3 - \alpha > 0$, so this lower bound diverges to $+\infty$ as $n \to \infty$. Therefore:
  \[
    \frac{T_{\text{Approx}}^{N_n}(n) - T_{\text{Gumbel}}(n)}{\sigma_g n^{1/3}} \xrightarrow{\mathbb{P}} +\infty
  \]

  Since the scaled $T_{\text{Gumbel}}(n)$ converges in distribution (hence is tight), the scaled $T_{\text{Approx}}^{N_n}(n)$ must diverge to $+\infty$ in probability.
\end{proof}
